% !TeX program = pdfLaTeX
\documentclass[12pt]{article}
\usepackage{amsmath}
\usepackage{graphicx,psfrag,epsf}
\usepackage{enumerate}
\usepackage{natbib}
\usepackage{textcomp}
\usepackage[hyphens]{url} % not crucial - just used below for the URL
\usepackage{hyperref}
\providecommand{\tightlist}{%
  \setlength{\itemsep}{0pt}\setlength{\parskip}{0pt}}

%\pdfminorversion=4
% NOTE: To produce blinded version, replace "0" with "1" below.
\newcommand{\blind}{0}

% DON'T change margins - should be 1 inch all around.
\addtolength{\oddsidemargin}{-.5in}%
\addtolength{\evensidemargin}{-.5in}%
\addtolength{\textwidth}{1in}%
\addtolength{\textheight}{1.3in}%
\addtolength{\topmargin}{-.8in}%

%% load any required packages here



% Pandoc citation processing


\begin{document}


\def\spacingset#1{\renewcommand{\baselinestretch}%
{#1}\small\normalsize} \spacingset{1}


%%%%%%%%%%%%%%%%%%%%%%%%%%%%%%%%%%%%%%%%%%%%%%%%%%%%%%%%%%%%%%%%%%%%%%%%%%%%%%

\if0\blind
{
  \title{\bf An Examination of Sport Climbing Scoring System}

  \author{
        Author 1 \thanks{Corresponding author: Name. Email:} \\
    \\
     and \\     Author 2 \\
    \\
     and \\     Author 3 \\
    \\
      }
  \maketitle
} \fi

\if1\blind
{
  \bigskip
  \bigskip
  \bigskip
  \begin{center}
    {\LARGE\bf An Examination of Sport Climbing Scoring System}
  \end{center}
  \medskip
} \fi

\bigskip
\begin{abstract}
The purpose of this paper is to investigate sport climbing, a new sport
featured in the 2020 Tokyo Summer Olympics. Simulation. Data Analysis:
drop and re-rank, correlations
\end{abstract}

\noindent%
{\it Keywords:} sport climbing, scoring system
\vfill

\newpage
\spacingset{1.45} % DON'T change the spacing!

\normalsize

\hypertarget{introduction-sport-climbing-other-scoring-system-mention-rank-product}{%
\section{Introduction (sport climbing, other scoring system, mention
rank
product)}\label{introduction-sport-climbing-other-scoring-system-mention-rank-product}}

In 2016, the International Olympic Committee announced the addition of
five new sports to the 2020 Summer Olympics in Tokyo, Japan, which would
then reschedule for 2021 due to the impact of the COVID-19 global
pandemic. The five new features to Tokyo 2020's competitions program
include baseball/softball, karate, skateboard, sports climbing, and
surfing. One of the new sports, sport climbing, got our attention,
specifically because of its unique scoring system and the fact that only
one set of medals is awarded for each gender.

Sport climbing at the 2020 Tokyo Olympics consists of three disciplines:
speed climbing, bouldering, and lead climbing. Speed climbing takes
place on a standardized course and competitors try to reach the top of
the course as fast as possible. For Tokyo 2020, speed climbing is being
contested in a head-to-head format with ranks determined by how far a
competitor advances in the bracket. In bouldering, contestants have a
fixed amount of time to complete as many courses as they can. Winners
are determined based on who completes the most courses and ties are
broken based on who had the fewest attempts. Ties are further broken by
the competitor achieved the most ``zone holds'', which are holds
approximately halfway through each course. Finally, in lead climbing, an
athlete gets one point for each hold that they reach, so whoever reaches
the highest point on the wall is the winner. Each lead climber only gets
one attempt and when they fall their attempt is over. These three
different climbing disciplines demand different sets of skills and,
often, athletes specialize in a single event. However, since only one
set of Olympic medals is awarded to sport climbing, rather than choosing
only one of these disciplines to include in the Olympics, all three
events were chosen to be included as a sort of climbing triathlon.

In the Olympic sport climbing, there are 20 competitors at the start (in
both men's and women's). All 20 competitors compete in each of the three
events in the qualification round, and their performances in each event
are ranked from 1 to 20. A competitors final score is then computed as
the product of their ranks in the three events and the lower product is
better. Specifically, \[Score_i = R^S_i\times R^B_i\times R^L_i,\] where
\(R^S_i\), \(R^B_i\), and \(R^L_i\) are the ranks of the \(i\)-th
competitor in speed climbing, bouldering, and lead climbing,
respectively.

The top 8 finishers in the qualification round advance to the finals
where they once again compete in all three events, they are again ranked
from 1 to 8, and their final score is the product of these three ranks
in the final. Whoever has the lowest product of ranks in the final wins
the gold medal. This type of scoring system heavily rewards high
finishes and relatively ignores poor finishes. For instance, if climber
A finished 1st, 20th, and 20th and climber B finished 10th, 10th, and
10th, climber B would have a score of 1000 whereas climber A would have
a much better score of 400, despite finishing last in 2 out of 3 of the
events.

Heavily criticized

Other sports scoring methods

\hypertarget{data-and-methods}{%
\section{Data and Methods}\label{data-and-methods}}

\hypertarget{data-collection}{%
\subsection{Data Collection}\label{data-collection}}

We collected data on major climbing competitions from 2018 to 2020,
including the 2020 Continental Championships of Europe, Africa, Oceania,
Pan-America; 2019 and 2018 World Championships; 2018 Asian Games; and
2018 Asian Games.

\hypertarget{eda}{%
\subsection{EDA}\label{eda}}

\hypertarget{results}{%
\section{Results}\label{results}}

\hypertarget{simulations}{%
\subsection{Simulations}\label{simulations}}

We conducted a simulation study to examine the rankings and scoring for
climbers in both qualification and final rounds. For each round, we
performed 10000 simulations, and this was accomplished by randomly
assigning the ranks of each event to every participant, with the
assumption that the ranks are uniformly distributed. After the
completion of the simulations, we calculated the final scores for every
simulated round, as well as the final standings for the climbing
athletes. This data would then enable us to answer questions about
various topics, including the distributions of scores for qualifying and
final rounds, and the probabilities of advancing to the finals or
winning a medal, given certain conditions.

\begin{figure}
\centering
\includegraphics{draft_files/figure-latex/unnamed-chunk-4-1.pdf}
\caption{Distribution of qualification scores}
\end{figure}

For the qualification round, a climber is almost guaranteed to make the
final round if they win the first event (with a 99.51\% chance of
advancing) or if they win at least one of the three climbing
concentrations (99.48\%). On the other hand, finishing last in the first
event or in any event would certainly hurt an athlete's chance of
finishing in the top 8, as the probabilities of a climber advancing
given they finish last in the first and in any event are 0.1830 and
0.1885, respectively. In addition, the average score for qualification
positions 1 to 8 are displayed in Table 1. We notice that on average,
the minimum score that one should aim for in order to move on to the
final round is 435 (for 8th rank).

\begin{table}[ht]
\centering
\caption{Average score for each qualifying rank} 
\begin{tabular}{rr}
  \hline
qual\_rank & avg\_adv\_score \\ 
  \hline
  1 & 36.02 \\ 
    2 & 73.61 \\ 
    3 & 115.40 \\ 
    4 & 162.23 \\ 
    5 & 216.00 \\ 
    6 & 278.16 \\ 
    7 & 350.33 \\ 
    8 & 434.59 \\ 
   \hline
\end{tabular}
\end{table}

Regarding the finals, a climber is very likely to earn a medal (finish
in the top 3) if they win the first event (83.03\% chance) or any event
(85.01\%). In order to obtain a climbing medal, the average score for
getting gold, silver, and bronze are 9.6748, 20.4143, and 33.2648,
respectively (Table 2).

\begin{table}[ht]
\centering
\caption{Average score of medalists} 
\begin{tabular}{rr}
  \hline
final\_rank & avg\_score \\ 
  \hline
  1 & 9.67 \\ 
    2 & 20.41 \\ 
    3 & 33.26 \\ 
    4 & 50.59 \\ 
    5 & 74.76 \\ 
    6 & 110.05 \\ 
    7 & 164.43 \\ 
    8 & 265.78 \\ 
   \hline
\end{tabular}
\end{table}

\begin{figure}
\centering
\includegraphics{draft_files/figure-latex/unnamed-chunk-7-1.pdf}
\caption{Boxplots of scoring distribution for every qualification rank}
\end{figure}

\hypertarget{data-analysis}{%
\subsection{Data Analysis}\label{data-analysis}}

\hypertarget{speed-climbing-vs-lead-and-bouldering}{%
\subsubsection{Speed climbing vs lead and
bouldering}\label{speed-climbing-vs-lead-and-bouldering}}

For our analysis on the relationship between the rankings of the events
and the final result, we used data from the 2018 Youth Olympics Women's
Qualification. Figure 3 is a scatterplot and correlation matrix between
the ranks of the individual events and the final standings, with
Kendall's Tau (Kendall Rank Correlation Coefficient) as our measure of
ordinal association between the quantities. It is evidently clear that
there is a strong and positive correlation between the ranks of
bouldering and lead climbing, and as a results, the standings of these
two events are highly correlated with the final rankings. On the other
hand, the correlation with the final rank is not as strong for speed
climbing. Thus, speed climbers are facing a huge disadvantage in this
scoring system, compared to those that are specialized in the other two
concentrations.

This trend also holds for most of the past competitions.

\begin{figure}
\centering
\includegraphics{draft_files/figure-latex/unnamed-chunk-9-1.pdf}
\caption{Kendall's rank correlations - 2018 World Championship, Women's
Qualification}
\end{figure}

\hypertarget{drop-and-re-rank}{%
\subsubsection{Drop and re-rank}\label{drop-and-re-rank}}

A single climber excluded changes things drastically, especially order
of medalists.

The cases where someone behind you drops out and your ranking changes.

Example from 2018 youth, women's final dropping ranks 3 and 5 change the
medalist order

\begin{figure}
\centering
\includegraphics{draft_files/figure-latex/unnamed-chunk-11-1.pdf}
\caption{Original rankings and rankings after each rank is dropped}
\end{figure}

\bibliographystyle{agsm}
\bibliography{}

\end{document}
